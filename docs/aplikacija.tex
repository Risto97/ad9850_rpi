\section{Aplikacija}

\subsection{Režimi rada}\label{working_modes}
Kao što je već rečeno \gls{cli} i \gls{gui} aplikacija trebaju da obezbede iste
funkcionalnosti:
\begin{itemize}
\item \label{run_mode} \textbf{Run} pokretanje generatora sa fiksnom frekvencijom bez vremenskog
  ograničenja rada, potreban argument:
  \begin{itemize}
    \item \textbf{freq} zadata frekvencija generisanog signala u Hz.
  \end{itemize}
\item \label{run_for_mode} \textbf{Run for} pokretanje generatora sa fiksnom frekvencijom sa
  određenim vremenom trajanja.
  \begin{itemize}
  \item \textbf{freq} zadata frekvencija generisanog signala u Hz.
  \item \textbf{time\_ms} trajanje generisanja signala u ms.
  \end{itemize}

\item \label{sweep_mode} \textbf{Sweep} linearna promena frekvencije između dve definisane
  vrednosti.
  \begin{itemize}
  \item \textbf{start\_freq} početna frekvencija u Hz.
  \item \textbf{stop\_freq}  krajnja frekvencija u Hz.
  \item \textbf{step\_freq}  vrednost koraka frekvencije u Hz.
  \item \textbf{step\_time}  vreme zadržavanja na svakom koraku u ms.
  \end{itemize}
  Ovaj režim rada generatora se često koristi u ispitivanju frekvencijske karakteristika
  elektronskih kola.
  Za prikazivanje frekvencijske karakteristike na osciloskopu pored signala promenljive
  frekvencije potrebno je obezbediti i trougaoni signal kao izvor vremenske baze
  osciloskopa. \\
  Ovakav uređaj se naziva Sweep Generator ili Wobbler.
  Često se jedan kanal osciloskopa može podesiti kao izvor vremenske baze dok
  drugi služi kao ulaz za Y-osu.

\end{itemize}

\subsubsection{Konfiguracija aplikacije} \label{config_file}
Potrebno je obezbediti mogućnost promene pinova preko kojih je \gls{rpi} povezan
sa AD9850 modulom bez ponovnog kompajliranja aplikacije. \\
Ovo je rešeno korišćenjem deljenog konfiguracionog fajla za \gls{cli} i
\gls{gui} aplikaciju. \\

Konfiguracioni fajl treba da sadrži sledeće parametre.
\begin{itemize}
\item \textbf{w\_clk} broj WiringPi pina povezanim sa \textbf{W\_CLK} pinom na AD9850 modulu.
\item \textbf{fq\_ud} broj WiringPi pina povezanim sa \textbf{FQ\_UD} pinom na AD9850 modulu.
\item \textbf{data} broj WiringPi pina povezanim sa \textbf{D7} pinom na AD9850 modulu.
\item \textbf{rst} broj WiringPi pina povezanim sa \textbf{RST} pinom na AD9850 modulu.
\item \textbf{dds\_clk} frekvencija referentnog oscilatora u Hz na AD9850 modulu.
\end{itemize}

Na GNU/Linux operativnim sistemima konfiguracioni fajlovi trebaju da se nalaze
u određenom direktorijumu. \\
Prema XDG Base Directory Specification\cite{xdg_spec} od freedesktop projekta
treba se poštovati sledeće pravilo za kreiranje konfiguracionog fajla.

\begin{itemize}
  \item Ukoliko je varijabla okruženja \textbf{XDG\_CONFIG\_HOME} postavljena,
    tada ona sadrži putanje direktrijuma gde se smeštaju korsnički
    konfiguracioni fajlovi za aplikacije.
  \item Ukoliko je varijabla okruženja \textbf{XDG\_CONFIG\_HOME} prazna kao
    direktorijum za smeštanje korisničkih konfiguracionih fajlova treba
    koristiti \textbf{\$HOME/.config}.
  \item Svaka aplikacija koja ima potrebe za konfiguracionim fajlom treba imati
    svoj direktorijum unutar \textbf{XDG\_CONFIG\_HOME} direktorijuma unutar kojeg će se
    nalaziti konfiguracioni fajlovi.
\end{itemize}

Ovo ne deluje kao previše bitna stavka, ali bi poštovanje ovih pravila trebalo
biti obavezno.
Zbog nepoznavanje ove specifikacije od strane programera nastali su programi
koji nemaju jedinstven direktorijum za konfiguracione fajlove, već se oni mogu naći
svuda po sistemu.
Ovo stvara problem prilikom backup-ovanja sistema, ukoliko su svi konfiguracioni
i korisnički fajlovi na definisanom mestu backup sistema može biti automatizovan i siguran. \\
Ova specifikacija definiše direktorijume i za ostale korisničke fajlove. \\

Potrebno je na siguran način parsirati i upisivati ove konfiguracione fajlove.
Kako bi se izbeglo ponovno pisanje ovakve biblioteke i izbegli rizici za ovaj projekat izabrana je \textbf{libconfig}\cite{libconfig} biblioteka.\\

Za pristup GPIO pinova RPi korišćena je WiringPi\cite{WiringPi} biblioteka

\subsection{CLI aplikacija}

\gls{cli} je aplikacija koja se izvršava bez grafičkog okruženja preko komandne linije.
Ovo je često bolji pristup za određene aplikacije.
Kod nekih aplikacija grafičko okruženje nije neophodno i često je
lakše i pouzdanije upravljati preko komandne linije. \\

\subsubsection{Prednosti CLI aplikacije}
CLI aplikacije su manje kompleksne od GUI aplikacija i
samim tim postoji manje mogućnosti za postojanje bug-ova.
Funkcionalnost programa je najčešće bitnija od grafičkog interfejsa, pisanje CLI
aplikacije zbog jednostavnosti omogućava programeru da se fokusira više na samu
funkcionalnost. \\

Dodatna prednost CLI aplikacije je lakše pokretanje i podešavanje aplikacije iz
skriptnih jezika (BASH, Python) ovo često kod GUI aplikacija nije moguće.
Skriptovanje omogućava automatizaciju poslova i uklanja potrebu za ljudskim faktorom.\\

\subsubsection{Parsiranje argumenata komandne linije}

Jedan od zadataka CLI aplikacije je parsiranje argumenata komandne linije.
Kako bi se skratilo vreme razvoja i izbegla mogućnost grešaka može se koristiti
neka od biblioteka za parsiranje argumenata, u ovom projektu je korišćena
\textbf{cxxxopts}\cite{cxxopts} biblioteka.

Ova biblioteka se sastoji samo od jednog Header fajla, time ukljanja potrebu za
kompajliranjem dodatne biblioteke.

\subsubsection{Uputstvo za upotrebu}

Pokretanjem \textbf{ad9850\_cli} izvršnog fajla sa argumentom \emph{---help} na
\textbf{stdout} će se ispisati uputstvo za upotrebu aplikacije. \\

\begin{lstlisting}[language=bash, caption=ad9850\_cli help opcija]
   $ ad9850 --help
   AD9850 cli application.
   Usage:
     AD9850 [OPTION...]

         --run arg        Run with frequency arg
         --run-for arg    Run with frequency arg[0] for time arg[1] in ms
         --sweep arg      Sweep from freq arg[0] to freq arg[1] with step freq
                          arg[2] and step time arg[3] in Hz and ms
         --stop           Stop ad9850
         --read-cfg       Prints tuple of configuration parameters to stdout
         --write-cfg arg  Write cfg, arg is tuple = w_clk,fq_ud,data,rst,dds_clk
         --help           Show help

\end{lstlisting} %$

Na osnovu uputstva možemo pokrenuti generator u Sweep režimu na sledeći način.

\begin{lstlisting}[language=bash, caption=ad9850\_cli sweep opcija]
  $ ad9850 --sweep 1000,10000,100,10
  Sweeping from 1000 Hz to 10000 Hz, with step 100 Hz and step time 10 ms
\end{lstlisting} %$
\subsection{GUI aplikacija}

\subsubsection{Uvod}
Napisana je pomoću Qt Framework-a \cite{qt}. \\
Za GUI aplikaciju potrebno je obezbediti intuitivan i jednostavan interfejs.
Potrebno je obezbediti pristup svim režimima rada navedenim u sekciji
\ref{working_modes}. \\
Pored toga treba obezbediti ispisivanje i promenu konfiguracionog fajla odnosno
promenu pinova kojom RPi pristupa AD9850 modulu i frekvenciju referentnog takta.

\subsubsection{Opis dizajna}
Za potrebe ovog projekta osmišljen je sledeći dizajn. \\
Glavni prozor aplikacije je podeljen na dva \textbf{Widget}-a \cite{QWidget}.
Levi Widget je klase \textbf{QTabWidget} koji podržava kartice. \\
Desni widget je fiksan i prikazuje statusne informacije, na njemu se još nalazi
i taster za zaustavljanje generatora. \\
Na glavnom prozoru se nalazi još i \textbf{Menu}\cite{QMenu}, preko kojeg se
može pokrenuti dijalog za pristup konfiguracionom fajlu koji je prikazan na
slici(\ref{config_dialog}). \\

Widget sa karticama trenutno sadrži Basic i Sweep karticu. Basic kartica
prikazana na slici(\ref{basic_tab}) pokriva
\textbf{Run} i \textbf{Run for} režime rada. \\
Sweep kartica pokriva prikazana na slici(\ref{sweep_tab}) pokriva \textbf{Sweep}
režim rada. \\

Organizacijom ovog widget-a pomoću kartica ostavlja se mogućnost proširenja programa
sa novim režimima rada, a da se pritom ne narušava dizajn interfejsa i izbegne
prenatrpanost. \\


\subsubsection{Basic kartica}

Kao što se vidi na slikama(\ref{basic_tab}, \ref{basic_tab2}), ova kartica se
sastoji iz dva polja za unos realizovana pomoću QSpinBox\cite{QSpinBox} klase i
dva tastera QPushButton\cite{QPushButton}. \\

Tasteri \textbf{Start} i \textbf{Run For} pokreću generator u režimima \textbf{Run} odnosno \textbf{Run For}. \\
U slučaju režima \textbf{Run} razmatra se samo polje za unos sa opisom
\textbf{Frequency}. \\
U slučaju režima \textbf{Run For} potrebno je uneti vremensko trajanje rada generatora u
polje za unos sa opisom \textbf{Run Time}. \\
\begin{figure}[H]
  \centering{
    \includegraphics[width=10cm]{img/AD9850_gui_run.png}
    \caption{AD9850 GUI Basic Tab}
    \label{basic_tab}
  }
\end{figure}

\begin{figure}[H]
  \centering{
    \includegraphics[width=10cm]{img/AD9850_gui_run_for.png}
    \caption{AD9850 GUI run for}
    \label{basic_tab2}
  }
\end{figure}

\subsubsection{Sweep kartica}

Na slici(\ref{sweep_tab}) prikazana je Sweep kartica, ona se sastoji iz četiri
polja za upis i tastera Sweep. \\
Opisi parametara u ovom režimu se nalaze u sekciji(\ref{working_modes})

\begin{figure}[H]
  \centering{
    \includegraphics[width=10cm]{img/AD9850_gui_sweep.png}
    \caption{AD9850 GUI Sweep Tab}
    \label{sweep_tab}
  }
\end{figure}

\subsubsection{Status widget}

Potrebno je da korsnik ima uvid u trenutno stanje generatora, u tu svrhu je
napravljen Status widget koji se nalazi sa desne strane interfejsa.\\
Može se videti na slikama(\ref{basic_tab}, \ref{basic_tab2}, \ref{sweep_tab}).
\\
Na njemu se nalazi QLabel\cite{QLabel} koji prikazuje trenutni režim u kojem
generator radi i njegove parametre, QLedIndicator\cite{QLedIndicator}
predstavlja indikator da li je generator aktivan ili nije i taster za stopiranje generatora.


\subsubsection{Dijalog za konfiguraciju}

Dijalogu za konfiguraciju može se pristupiti preko config menija na glavnom
prozoru, prikazan je na slici(\ref{config_dialog}). \\
Sastoji se od četiri \textbf{QComboBox}\cite{QComboBox} padajućih prozora u
kojima se nalaze brojevi dostupnih pinova prema \textbf{WiringPi} biblioteci, jednog
\textbf{QSpinBox} polja za unos frekvencije referentnog oscilatora i dva tastera. \\

Prilikom pokretanja ovog dijaloga čita se konfiguracioni fajl opisan u
sekciji(\ref{config_file}), zatim prikazuje pročitani parametar u odgovarajuće
polje. \\

Nakon promene vrednosti padajućih prozora ili polja za unos moguće je sačuvati
promene u konfiguracioni fajl pritiskom na tastere \textbf{Save config}. \\
Pritiskom na taster \textbf{Ok} aplikacija učitava novo podešene parametre. \\

\begin{figure}[H]
  \centering{
    \includegraphics[width=6cm]{img/AD9850_gui_cfg.png}
    \caption{AD9850 konfiguracioni dijalog}
    \label{config_dialog}
  }
\end{figure}